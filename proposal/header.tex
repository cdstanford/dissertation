%%% Header for dissertation proposal + dissertation
%%% Caleb Stanford
%%% January 2021 -- Present

%% ========== General ==========

\usepackage[margin=1.5in]{geometry}
\usepackage[dvipsnames]{xcolor}

\usepackage{amsmath,amssymb,amsthm}
\theoremstyle{definition}
\newtheorem{definition}{Definition}
\newtheorem{example}{Example}
\newtheorem{proposition}{Proposition}
\newtheorem{theorem}{Theorem}

\usepackage{hyperref}
\hypersetup{
    colorlinks,
    linkcolor={red!50!black},
    citecolor={blue!50!black},
    urlcolor={blue!80!black}
}
\usepackage{cleveref}
\usepackage{multibib}
\newcites{Main}{Primary References}

%% ========== Graphics ==========

\usepackage{graphicx}
\usepackage{graphbox} % allows using [align=c] argument to center vertically
\usepackage{stmaryrd} % Symbols like \llbracket, \rrbracket
\usepackage{pifont} % Symbols like \ding

\newcommand{\GreenYes}{\color{ForestGreen}{Yes}}
\newcommand{\RedNo}{\color{red}{No}}
\newcommand{\cmark}{\ding{51}}
\newcommand{\xmark}{\ding{55}}

%% ========== Arrays / Tables ==========

\usepackage{verbatim}
\usepackage{algorithm, caption}
\usepackage[noend]{algpseudocode} % (layout for algorithmicx)
\usepackage{subcaption}
\captionsetup{compatibility=false} % Fix compatibility error
\usepackage{listings}

\usepackage{array}
\usepackage{makecell} % Line breaks in tables
% Line break in table cell
% \newcommand{\specialcell}[2][c]{%
% \begin{tabular}[#1]{@{}c@{}}#2\end{tabular}}
% Left/center/right aligned paragraph columns:
\newcolumntype{L}[1]{>{\raggedright\let\newline\\\arraybackslash\hspace{0pt}}m{#1}}
\newcolumntype{C}[1]{>{\centering\let\newline\\\arraybackslash\hspace{0pt}}m{#1}}
\newcolumntype{R}[1]{>{\raggedleft\let\newline\\\arraybackslash\hspace{0pt}}m{#1}}

\newenvironment{Tabular}[2][1] % stretchable tabular
  {\def\arraystretch{#1}\tabular{#2}}
  {\endtabular}

% stack text on top of other text
% usage: \stackanchor{A}{B} or \Shortstack{{a} {b} {c}}
\usepackage{stackengine}

%% ========== TikZ ==========

\usepackage{tikz}
\usetikzlibrary{positioning, matrix, fit, arrows, shapes}
\usetikzlibrary{fit}

%% Block Diagrams
\tikzset{Block/.style={draw, rectangle, inner sep=4pt, align=center}}
\tikzset{Block Edge/.style={draw,thick,->}}
% \tikzset{Data/.append style={ellipse}}
% \tikzset{Input/.append style={fill=red!20}}
% \tikzset{Output/.append style={fill=blue!20}}
\tikzset{Data/.append style={fill=blue!20}}
\tikzset{Input/.append style={}}
\tikzset{Output/.append style={}}

%% Dataflow Graphs
\tikzset{source/.style={
        draw,
        rounded rectangle,
        inner sep=2pt,
        align=center,
        fill=blue!20
}}
\tikzset{operator/.style={
        draw,
        rectangle,
        inner sep=2pt,
        align=center
}}
\tikzset{sink/.style={
        draw,
        rounded rectangle,
        inner sep=2pt,
        align=center,
        fill=blue!20
}}
\tikzset{dataflowedge/.style={draw,thick,->}}

%% Topology Diagrams
\tikzset{Device Node/.style={draw,circle}}
\tikzset{Network/.style={draw,cloud,cloud puffs=10,cloud puff arc=120, aspect=2.5, inner sep=2pt}}
\tikzset{In Edge/.style={->,very thick,red}}
\tikzset{Out Edge/.style={->,very thick,blue}}

% Dependency Relations
\tikzset{Tag Node/.style={draw, circle, inner sep=1pt}}
\tikzset{Tag Edge/.style={draw, thick}}
\tikzset{Tag Loop/.style={draw, thick, in=245, out=305, looseness=5}}
\newcommand{\KeyDepGraph}[1]{
    \begin{tikzpicture}[baseline=(1.base)]
        \node[Tag Node] (1) at (0,0) {\tg{#1}$_1$};
        \draw (1) edge[Tag Loop] (1);
        \node[Tag Node] (2) at (1.2,0) {\tg{#1}$_2$};
        \node[draw=none,fill=none] at (2.1,0) {$\cdots$};
        \draw (2) edge[Tag Loop] (2);
        \node[Tag Node] (3) at (3,0) {\tg{#1}$_k$};
        \node[draw=none,fill=none] at (3.9,0) {$\cdots$};
        \draw (3) edge[Tag Loop] (3);
    \end{tikzpicture}}
\newcommand{\ExtendedKeyDepGraph}[1]{
    \begin{tikzpicture}[baseline=(1.base)]
        \node[Tag Node] (1) at (0,0) {\tg{#1}$_1$};
        \draw (1) edge[Tag Loop] (1);
        \node[Tag Node] (2) at (1.2,0) {\tg{#1}$_2$};
        \node[draw=none,fill=none] at (2.1,0) {$\cdots$};
        \draw (2) edge[Tag Loop] (2);
        \node[Tag Node] (3) at (3,0) {\tg{#1}$_k$};
        \node[draw=none,fill=none] at (3.9,0) {$\cdots$};
        \draw (3) edge[Tag Loop] (3);
        \node[Tag Node] (4) at (1.6,1.2) {\tg{EOD}};
        \draw (4) edge[Tag Loop, in=65, out=115] (4);
        \draw (1) edge[Tag Edge] (4);
        \draw (2) edge[Tag Edge] (4);
        \draw (3) edge[Tag Edge] (4);
        \node[Tag Node] (5) at (0,1.2) {\tg{EOM}};
        \draw (5) edge[Tag Loop, in=65, out=115] (5);
        \draw (4) edge[Tag Edge] (5);
    \end{tikzpicture}}

%% ========== Code and Pseudocode Customization ==========

%% Listings
\lstset{language=Java,
  showspaces=false,
  showtabs=false,
  breaklines=true,
  showstringspaces=false,
  breakatwhitespace=true,
  commentstyle=\color{gray},
  keywordstyle=\color{blue},
  stringstyle=\color{red},
  basicstyle=\ttfamily\footnotesize,
  frame = single,
  numbersep = 0,
  resetmargins = true
}
\newcommand{\inljava}[1]{\lstinline[columns=fixed, basicstyle=\ttfamily\small]{#1}}

%% Algorithmic
\renewcommand{\algorithmicrequire}{\textbf{Input:}}
\renewcommand{\algorithmicensure}{\textbf{Output:}}
\algblockdefx[OnElement]{OnElement}{EndOnElement}%
  [2]{\textbf{on new element }#1\textbf{ in stream }#2\textbf{:}}%
  {\textbf{done}}
\algblockdefx[OnEmptyStreams]{OnEmptyStreams}{EndOnEmptyStreams}%
  {\textbf{on both streams empty:}}%
  {\textbf{done}}
\makeatletter
\ifthenelse{\equal{\ALG@noend}{t}}%
  {\algtext*{EndOnElement} \algtext*{EndOnEmptyStreams}}%
  {}%
\makeatother
