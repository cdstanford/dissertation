\section{Discussion}
\label{sec:discussion}

% \section{Directions}
% \label{sec:directions}

In the remainder of the thesis, we plan the following two research directions spanning approximately a 6-month period.
Much of this research follows an implementation of the ideas laid out in sections 3 and 4 of our recent paper~\cite{pods21}.
We plan to target machine learning and data analytics as target applications.

\subsection{Implementation of synchronization schema types}

We plan to implement synchronization schemas as a typing abstraction in Timely Dataflow~\cite{Timely} in Rust.
The goal of this implementation is to offer a type-safe programming abstraction on top of the low-level dataflow model which guarantees ordering of events in each stream.
Type safety conditions are partially checked statically, and partially compiled
to external \emph{verification conditions} checked by an SMT solver.

Here is a sketch of the proposed solution:
we write a library in which operators can be defined, where each operator has an input synchronization schema $I$ and output synchronization schema $O$,
and these can then be composed as a dataflow graph.
For each operator, it is compiled to a back-end Timely Dataflow operator,
which is implemented as a partitioned state update function;
in the Timely representation, we also indicate partitioning of events consistent
with the synchronization schema.
Then, for each composition (edge in the dataflow graph),
we generate a verification condition that states whether the output synchronization schema of each operator implies the input synchronization schema of the next operator.

\subsection{Implementation of the execution model}

Second, we plan to leverage the state-machine representation of data transducers as part of this framework.
Here is how that would look: part of the library allows writing
a query using a distributed variant of quantitative regular expressions.
The query is then compiled to Timely with an input and output synchronization schema derived from the query as part of the dataflow representation above.

While this allows for ordering guarantees, one question is how to then derive performance guarantees for the operator.
We propose a combination of empirical and analytical information: using the state machine representation we can calculate an analytical space and time bound for processing items,
and then using empirical testing we can derive running time bounds for the operators in the graph.
Another approach would be to define an abstract notion of latency and throughput in number of steps only (without the empirical component),
and use this to derive logical verification conditions related to number of state updates required to process a single input item for each operator (latency),
and number of state updates that can be processed in parallel (throughput).

\subsection{Other ideas}

We do not propose these for the thesis,
but we are also working on an implementation of an optimization framework
for database queries over smart watches for optimizing energy use.
Another interesting direction is how to incorporate edge computing metrics into the stream processing framework; some ideas are laid out in~\citeMain{wpe2}.

% TODO incorporate
% There are many directions for future work (\Cref{sec:discussion} and \Cref{sec:fw}). We are working on a partial implementation of the synchronization schemas~\citeMain{pods21} and data transducer~\citeMain{popl19} abstractions on top of Timely Dataflow~\cite{Timely,Naiad2013} in Rust~\cite{RustLang}.
% Timely is a good choice because it offers a semantically sound low-level dataflow representation,
% and we aim to leverage Rust's type system for compile-time guarantees,
% while generating external verification conditions to prove user programs correct.
% More broadly, verifying properties of streaming programs is a major future direction.
