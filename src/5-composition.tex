\chapter{Compositionality}
\label{cha:composition}
\headerblock{%
  \headerquote{The meaning of a compound expression is a function of the meanings of its parts and of the syntactic rule by which they are combined.}{Barbara Partee, 1984~\cite{partee1984compositionality}, as formulated in~\cite{partee1990mathematical}}
  % Partee 1984: The meaning of an expression is a function of the meanings of its parts and of the way they are syntactically combined
  % Partee et. al. 1990: The meaning of a compound expression is a function of the meanings of its parts and of the syntactic rule by which they are combined
}

In this section, we define \emph{SPS-transformers} (SPSTs), a programming language for computations over series-parallel streams.
If synchronization schemas are provided as types for the input and output streams of a computation,
then an SPST is a (type-safe and deterministic) function from input batches to output batches, which respects the structure given by the types. This material was originally presented in~\citeMain{pods21} (some of it was delegated to the appendix).

The primary goal of SPSTs is to be \emph{compositional}: they allow for defining operators over a stream in a compositional way with respect to the type of that stream.
% This is similar to how events, batches, linearizations, and other relations in \Cref{cha:foundation} are defined compositionally on the structure of the stream via typing rules.
Per the goals of our introduction, every SPST should satisfy a type safety property.
Formally, it should be a deterministic function from input batches to output batches.
By \Cref{thm:type-safe-and-deterministic-converse} this implies that a corresponding \emph{operator}, a low-level function from lists of input to lists of output, can be defined that is type-safe and deterministic
in a more realistic event-by-event processing sense.
We will define the semantics of an SPST inductively on the input structure; the typing judgments for this semantics in each case demonstrate type safety.

An interesting aspect of the material in this section -- which we have found helpful as a way to think about future work in this space -- is the approach to achieve incremental processing by distinguishing \emph{open} and \emph{closed} semantics. For each transformer, the open semantics represents a monotonically-increasing function from input streams to output streams, i.e., an operator that satisfies \emph{monotonicity}. The \emph{closed} semantics represents the semantics after the input stream is terminated, so that the stream operator may produce some final output that is not incremental.
For example, using this distinction, the function \textsc{Sum} described in \Cref{fig:operator-properties-examples} can be represented as a function which produces no output in its open semantics, but produces the total sum in its closed semantics.
We describe this distinction further in \Cref{45:sec:design-goals}.

\section{Notational Differences}

The material in this section uses the definitions, notation, and terminology from~\citeMain{pods21},
rather than those from \Cref{cha:foundation}.
Stream types are called \emph{synchronization schemas},
and batches are called \emph{series-parallel streams}, or SPSs
for short.
In synchronization schemas, instead of base types $T$, there are sets of headers $\mathcal{H}$, representing the possible tuples that might occur for an event
(See \Cref{sec:historical}).
So if $T = \mathcal{H} = (H_1, H_2, H_3)$ for example, then $H_1$, $H_2$, and $H_3$ are tuple types and the events of type $T$ are either tuples of type $H_1$, or tuples of type $H_2$, or tuples of type $H_3$.
It uses the notation $\preceq$ for stream prefix.
This is defined using concatenation:
for batches $b_1$ and $b_2$, we say $b_1 \preceq b_2$ if there is some $b_1'$ such that $b_2 = b_1 \circ b_1'$.

\section{Design Goals}
\label{45:sec:design-goals}

In addition to satisfying \textbf{monotonicity}, \textbf{type safety}, and \textbf{determinism} as already discussed,
SPSTs should satisfy the following additional design goals.
First, they should respect the \textbf{parallelism}
in the input and output: parallel input events should be processed in parallel,
and parallel input threads should produce parallel output events.
% Thus, SPS transformers should be guaranteed to respect and maintain that parallelism.
For example, given an input which is two streams in parallel, the computation should be written in such a way that the two streams are processed separately, and outputs corresponding to them should be unordered.
Second,
to allow for the specification of potentially complex computations, we additionally want our language to be \textbf{compositional}: it should be natural to construct a computation by combining sub-computations.
For example, processing a stream of the hierarchical type $\hier{\mathcal{H}}{S}$
should be definable, both syntactically and semantically, in terms of
existing computations defined over sub-streams of type $S$.
Finally, any computation in our language should be \textbf{incremental} (or \emph{streamable}): it should process the input in one pass, producing output incrementally.

To satisfy these design goals, we make the following technical choices.
First, to satisfy the parallelism goal, we define SPSTs to have SPSs as input and output rather than sequential objects.
The input being an SPS allows us to specify the computation to exploit parallelism,
and the output being an SPS requires that we respect parallelism when producing output events.

To understand  the challenges in defining the semantics due to the interplay between streamability and compositionality, consider a transformer $P$ processing
hierarchical streams of type  $S=\hier{\mathcal{H}}{S'}$ that we would like
define in terms of a transformer $P'$ processing streams of type $S'$.
Consider an input stream $t=[\spsemp, d, t']$ of type $S$ for an $\mathcal{H}$-tuple $d$ and
stream $t'$ of type $S'$. Suppose we want to extend the input stream $t$ with
a tuple $d'$. If the type of $d'$ is one of the headers appearing in $S'$, then
it really extends the sub-stream $t'$, and should be processed by the transformer $P'$.
For incrementality, we want to make sure that, while processing $d'$, $P'$ extends the output stream only
by adding new items. Formally, this means that the output of $P'$ on the input
stream $t'$ should be a prefix of its output on the stream $t'\circ d$.
With this motivation, we define such a semantics, which we call \emph{open} semantics,
for transformers as functions from input to output streams, and ensure that it is \emph{monotonic}
 with respect to prefix ordering
(see Theorem~\ref{45:thm:spst-monotonicity}).
But now suppose that the item $d'$ is an ${\mathcal H}$-tuple
that acts as a synchronization marker for the events in the sub-stream $t'$.
Then to process it, the transformer $P'$ should return, and let the top-level
transformer $P$ process the item $d'$.
During this return, the transformer $P'$ can do additional computation and produce
additional output items even though the stream it has processed is still $t'$.
This is a typical case when $S'$ corresponds to key-based partitioning,
and the arrival of the synchronization marker $d'$ triggers the reduce operation
that aggregates the results of the computations of the key-indexed sub-streams of $t'$.
This though requires us to define another semantics of the transformer $P'$ on
the input stream $t'$ that extends the open semantics and includes the results
of the computation upon return. We call it \emph{closed} semantics to indicate
that it is applicable when the current stream  is being closed.
Note that the result of computation of $P$ on the stream $[\spsemp, d,t',d', \spsemp]$ can be
described by relying on the closed semantics of $P'$ on the stream $t'$.
In terms of existing work on punctuation,
the closed semantics can be thought of as the stream output on
a stream terminated by an \emph{end-of-stream} marker.

Finally, an SPST is a function on pairs:
it takes an initial value and an input
SPS to a final value and an output SPS.
We need this for compositionality: without the initial value as input, an SPS-transformer
on a sub-stream of the input could not be initialized based on the surrounding context.
Similarly, the final value (separate from the series of output items produced) can be
used to describe a summary of the input stream to be used in the surrounding context
when the computation finishes.

%this means that if the input stream is extended (additional items added that are later than all existing items), the output must also be an extension of the previous output. However, to satisfy compositionality, sometimes for an \emph{intermediate} result we don't need it to be monotonic in order for the overall computation to be monotonic. To allow for this, formally, we distinguish between two semantics: an monotonic \emph{open} semantics, which takes an input stream which is thought of as being unfinished (may receive more items), and produces an output stream; and a \emph{closed} semantics, which takes an input stream which is thought of as being finished (may not receive more items), and produces a final output, which need not be monotonic. The open semantics will be a prefix of the closed semantics. The final value of the computation is naturally a part of the closed semantics, but not the open semantics.

We summarize all of these choices in the following definition of the \emph{interface}
for an SPST.
We also define subtyping for the interface, where the output is relaxed.
Each of the language constructs will then implement this interface.
In \Cref{45:ssec:spst-example}, we give an extended example
to illustrate the formal definitions in this section.

\begin{definition}[SPS-transformer interface]
An SPS-transformer (SPST) $P$
has:
\begin{itemize}
\item A \emph{type} denoted $(X, S, X', S')$,
where
$X$ is the type for the initialization value,
$S$ is an input synchronization schema,
$X'$ is a type for the final return value,
and $S'$ is an output synchronization schema.
We write $P: (X, S, X', S')$.

\item An \emph{open semantics} denoted $\spstsemO{x}{t}{P} = t'$,
where $x: X$ is the initial value, $t: S$ is the input SPS,
and $t': S'$ is the incrementally produced output SPS.

\item A \emph{closed semantics} denoted $\spstsemC{x}{t}{P} = (x', t')$,
where $x': X'$ is the initial value, $t: S$ is the input SPS,
$x': X'$ is the final value, and $t': S'$ is the output SPS.
We additionally enforce that the open semantics is a prefix
of the closed semantics:
$\spstsemO{x}{t}{P} \prefix t'$.

\end{itemize}
\end{definition}

\begin{definition}
\label{45:def:spst-subtyping}
If $S_1' \lesssim S_2'$ (Definition~\ref{def:stream-relaxation}),
then $(X, S, X', S_1')$ is a \emph{subtype} of $(X, S, X', S_2')$.
If $P: (X, S, X', S_1')$
then we also write $P: (X, S, X', S_2')$.
The open and closed semantics are derived as the unique output stream
given by Proposition~\ref{45:prop:schema-relaxation-flattening}.
\end{definition}

\noindent
In the remainder of the section,
we give one language construct corresponding to each constructor of the input SPS.
Some additional notation:
for a set of headers $\mathcal{H}$,
we write $\tup(\mathcal{H})$ for the set of tuples $x: \mathcal{H}$.
For a synchronization schema $S$,
we write $\sps(S)$ for the set of SPSs $t: S$.
We write $\bag(X)$ for the set of bags (multisets) of items of type $X$.
% \str
% \caleb{\bag(X) is currently only used in the partition-by case. I wonder if we can replace
% it with sps(Bag(..)) for some appropriate schema.}

% \kk{Rajeev suggestions:}
% \kk{Make relational SPST a special case and don't even define syntax.}
% \kk{Closed semantics should be renamed and only return the extension of the sps and the final value. (Maybe)}

\section{Syntax and Semantics}

\subsection{Relational SPST}

We start with the relational SPST, which represents a standard relational operator that can be used to process a bag of items, producing another bag of items.
Relational operators are well studied and are commonly defined using SQL
and its extensions.
Our design choice here is to not impose a particular relational base language
or SQL variant;
instead,
the relational operator is given as two \emph{black-box} functions,
which define the open and closed semantics,
respectively.
We only require that these are functions on bags (i.e., independent of the input order),
and that the open semantics is monotone and a prefix
of the closed semantics.

\begin{definition}[Relational SPST]
A relational SPST
\[P : (X, \relleaf{\mathcal{H}}, X', \relleaf{\mathcal{H}'})\]
consists of two fields:
\begin{align*}
\spstfield{P}{open} &: X \times \sps(\relleaf{\mathcal{H}}) \to \sps(\relleaf{\mathcal{H}'}) \\
\text{and} \quad
\spstfield{P}{closed} &: X \times \sps(\relleaf{\mathcal{H}}) \to X' \times \sps(\relleaf{\mathcal{H}'}).
\end{align*}
such that
(1) $\spstfield{P}{open}$ is \emph{monotone}:
if $r_1 \prefix r_2$, then $\spstfield{P}{open}(x, r_1) \prefix \spstfield{P}{open}(x, r_2)$;
and
(2) $\spstfield{P}{open}$ is a prefix of $\spstfield{P}{closed}$:
if $\spstfield{P}{closed}(x, r)$ $= (x', r')$
then $\spstfield{P}{open}(x, r) \prefix r'$.
The semantics of $P$ is defined as
$\spstsemO{x}{r}{P} = \spstfield{P}{open}(x, r)$
and $\spstsemC{x}{r}{P} = \spstfield{P}{closed}(x, r)$.

\end{definition}

\subsection{Parallel SPST}

We now define the inductive SPSTs.
An SPST processing inputs of type
$\parcomp{S_1}{S_2}$ is composed of two SPSTs running in parallel independently.
The question here is, can the components SPSTs produce tuples of the same type?
The answer is yes, provided such tuples, since they get produced independently,
are summarized using a schema $\relleaf{\mathcal{O}}$,
where $\mathcal{O}$ is a set of output headers.
So the output schema for the parallel SPST
will be $\parthree{S_1'}{S_2'}{\relleaf{\mathcal{O}}}$.
%The semantic behavior of a parallel SPST respects the parallelism in the input: it %first initializes the states of the two sub-SPSTs, it then runs the two sub-SPSTs in parallel.
% For the closed semantics, we also can process the two final values of the sub-SPSTs.

\begin{definition}[Parallel SPST]
Let $S_1, S_2, S_1', S_2'$ be schemas.
A parallel SPST
\[
P : (X, \parcomp{S_1}{S_2}, X', \parthree{S_1'}{S_2'}{\relleaf{\mathcal{O}'}})
\]
consists of internal types $X_1, X_2, X_1', X_2'$ and
four fields:
\begin{align*}
\spstfield{P}{left} &: (X_1, S_1, X_1', \parcomp{S_1'}{\relleaf{\mathcal{O}'}}), \\
\spstfield{P}{right} &: (X_2, S_2, X_2', \parcomp{S_2'}{\relleaf{\mathcal{O}'}}), \\
\spstfield{P}{init} &: X \to X_1 \times X_2,
\quad \text{and} \quad
\spstfield{P}{fin} : X_1' \times X_2' \to X'.
\end{align*}
The semantics of $P$ is as follows: if we have that
$\spstfield{P}{init}(x) = (x_1, x_2)$,
$\spstsemO{x_1}{t_1}{\spstfield{P}{left}} = \spspar{t_1'}{r_1'}$,
and
$\spstsemO{x_2}{t_2}{\spstfield{P}{right}} = \spspar{t_2'}{r_2'}$,
where $r_1', r_2': \relleaf{\mathcal{O}'}$,
and
additionally $\spstsemC{x_1}{t_1}{\spstfield{P}{left}} = (x_1', \spspar{t_1''}{r_1''})$
and
$\spstsemC{x_2}{t_2}{\spstfield{P}{right}} = (x_2', \spspar{t_2''}{r_2''})$,
then
\begin{align*}
\spstsemO{x_1}{t_1}{P}
    &= \spspar{\spspar{t_1'}{t_2'}}{r_1' \cup r_2'} \\
\spstsemC{x_1}{t_1}{P}
    &= ( \spstfield{P}{fin}(x_1', x_2'),
         \spspar{\spspar{t_1''}{t_2''}}{r_1'' \cup r_2''} ).
\end{align*}
\end{definition}

\subsection{Hierarchical SPST}

When the input schema is $S=\hier{\mathcal{H}}{S_1}$, we want to
define the corresponding SPST $P$ parameterized by a sub-SPST from
$S_1$ to $S'_1$.
The SPST $P$ maintains its own state that gets updated sequentially
whenever any $\mathcal{H}$-tuple is processed,
is passed to the sub-SPST when called, and is updated when the sub-SPST returns.
The output schema of $P$ has the same structure as the input:
it is divided into synchronizing events and non-synchronizing events.
On input synchronization events, any output tuple may be produced,
including a synchronization event; but on input sub-stream events, it would be incorrect
to produce an output synchronizing event, as this would not be produced in a consistent order.
The distinction between closed and open
semantics plays a key role here:
synchronizing events, when processed by $P$, ``close'' the computation of the sub-SPST.
To formalize this inductively,
we introduce an auxiliary semantics $\spstsemI{y}{t}{P}$
where the output is an internal state (rather than a final value),
and in which the input stream ends with a $d_i$ event,
i.e., the final $t_i$ is $\spsemp{}$.

\begin{definition}[Hierarchical SPST]
Let $S_1$ and $S_1'$ be schemas, and $\mathcal{H}$ and $\mathcal{H}'$
be a set of input and output headers, respectively.
Let $S' = \hier{\mathcal{H}'}{S_1'}$.
A hierarchical SPST
\[
P : (X, \hier{\mathcal{H}}{S_1}, X', \hier{\mathcal{H}'}{S_1'})
\]
consists of internal types $X_1, X_1', Y$ and
six fields:
\begin{alignat*}{3}
\spstfield{P}{sub} &: (X_1, S_1, X_1', S_1'),
    \hspace{-1em} \\
\spstfield{P}{update} &: Y \times \tup(\mathcal{H}) \to Y \times \sps(S'),
    \hspace{-8em} \\
\spstfield{P}{call} &: Y \to X_1,
    &\spstfield{P}{return} &: Y \times X_1' \to Y, \\
\spstfield{P}{init} &: X \to Y,
    &\text{and} \quad \spstfield{P}{fin} &: Y \to X' \times \sps(S').
\end{alignat*}

The auxiliary semantics of $P$ is denoted
$\spstsemI{y}{t}{P} = (y', t')$, where $y, y': Y$,
and defined inductively
\emph{only} for $t$
of the form $[t_0, d_1, t_1, \ldots, d_m, \spsemp{}]$.
For the base case, $\spstsemI{y}{\spsemp{}}{P} = (y, \spsemp{})$.
Then inductively, if
$\spstsemI{y}{t}{P} = (y', t')$,
$t_1: S_1$, and
$d: \mathcal{H}$,
and if we have
$\spstfield{P}{call}(y') = x_1$,
$\spstsemC{x_1}{t_1}{\spstfield{P}{sub}} = (x_1', t_1')$,
$\spstfield{P}{return}(y', x_1') = y''$,
and $\spstfield{P}{update}(y'', d) = (y''', t'')$,
then
$\spstsemI{y}{t \circ \spsseq{t_1, d, \spsemp}}{P} = (y''', t' \circ t_1' \circ t'')$.
Given the auxiliary semantics, we define the semantics
of $P$ on a trace decomposed as $t \circ \spsseq{t_1}$,
where the last list item of $t$ is an empty sub-trace.
Let
$\spstfield{P}{init}(x) = y$,
$\spstsemI{y}{t}{P} = (y', t')$,
and $\spstfield{P}{call}(y') = x_1$.
Additionally, let
$\spstsemC{x_1}{t_1}{\spstfield{P}{sub}} = (x_1', t_1')$,
$\spstfield{P}{return}(y', x_1') = y''$,
and $\spstfield{P}{fin}(y'') = (x', t'')$.
Then:
\begin{align*}
\spstsemO{x}{t}{P}
    &= t' \circ \spstsemO{x_1}{t_1}{\spstfield{P}{sub}} \\
\spstsemC{x}{t}{P}
    &= (x', t' \circ t_1' \circ t'').
\end{align*}
\end{definition}

% \subsection{Sequential SPST}

% Recall that $\seqleaf{\mathcal{H}}$ is a useful special case of hierarchical synchronization schemas
% that denotes simple sequences, i.e., it is the schema $\hier{\mathcal{H}}{\relleaf{\varnothing}}$.
% We define an SPST for this case for convenience,
% a special case of SPST for the hierarchical case.
% This SPST transform a sequence of tuples and an initial state to an output sequence of tuples and a final state.
% Sequential SPSTs are used for any computation that depends on the order of the input tuples, such as a linear interpolation, or string recognition.

% \begin{definition}[Sequential SPST]
% A sequential SPST
% \[
% P : (X, \seqleaf{\mathcal{H}}, X',  \seqleaf{\mathcal{H'}})
% \]
% consists of four fields: an internal state type $Y$, an initialization function $\spstfield{P}{init} : X \to Y$, a sequential update function $\spstfield{P}{update} : Y \times \tup(\mathcal{H}) \to Y \times (\mathcal{H'})^{*}$, and a finalization function $\spstfield{P}{fin} : Y \to X'$.
% Semantically, it functions the same as the following hierarchical SPST $P'$:
% $P'$ has the same internal state type $Y$;
% the same $\spstfield{P'}{init}$, $\spstfield{P'}{update}$, and $\spstfield{P'}{fin}$;
% a child SPST $\spstfield{P'}{sub}: (Y, \relleaf{\varnothing}, Y, \relleaf{\varnothing})$
% which is the trivial relational operator
% i.e., $\spstfield{(\spstfield{P'}{sub})}{op}(y, ()) = (y, ())$,
% and the identity call and return functions
% $\spstfield{P'}{call}: Y \to Y$ and $\spstfield{P'}{return}: Y \to Y$.
% \end{definition}

\subsection{Partitioned SPST}

Finally, we define SPST for the partition-by case.
The idea here is analogous to the parallel composition $\parcomp{S_1}{S_2}$ case: each sub-stream corresponding to a different key value may produce output corresponding to that key value, \emph{or} produce output corresponding to a common bag of tuples $\mathcal{O}'$.
The partitioned SPST initializes the state of $\spstfield{P}{sub}$ for each key with a nonempty SPS and runs the child SPST for each (non-empty) key in parallel.
We additionally need an aggregation stage (applicable to the closed semantics only),
in which we combine all of the partitioned states
using a black-box relational operator $\spstfield{P}{agg}$,
similar to what was done in the relational SPST base case.

\begin{definition}[Partitioned SPST]
Let $S = \keyby{K}{S_1}$ and $S' = \keyby{K}{S_1'}$ be schemas,
and $\mathcal{O}'$ a set of headers.
A partitioned SPST
\[
P : (X, \keyby{K}{S_1}, X', \parcomp{\keyby{K}{S_1'}}{\relleaf{\mathcal{O}'}}
\]
consists of internal types $X_1, X_1'$ and three fields:
\begin{align*}
\spstfield{P}{sub} &: (X_1, S_1, X_1', \parcomp{S_1'}{\relleaf{\mathcal{O}'}}), \\
\spstfield{P}{init} &: X \times \tup(K) \to X_1, \\
\text{and} \quad
\spstfield{P}{agg} &: X \times \bag((\tup(K) \times X_1') \to X' \times \bag(\tup(\mathcal{O}')).
\end{align*}

For the semantics,
suppose $t = \spsbag{\keyed{v_1}{t_1}, \ldots, \keyed{v_m}{t_m}}$,
and for $i = 1, \ldots, m$,
$\spstfield{P}{init}(x, v_i) = x_i$,
$\spstsemC{x_i}{t_i}{\spstfield{P}{sub}} = (x_i', \spspar{t_i'}{r_i'})$,
$\spstfield{P}{agg}(x, \{(v_1, x_1), \ldots, (v_m, x_m)\}) = (x', r_0')$,
and
$\spstsemO{x_i}{t_i}{\spstfield{P}{sub}} = \spspar{t_i''}{r_i''}$.
Then
\begin{align*}
    \spstsemC{x}{t}{P}
    &= (x', \spspar{ \spsbag{\keyed{v_1}{t_1'}, \ldots, \keyed{v_m}{t_m'}}}{r_0' \cup r_1' \cup \cdots \cup r_m'}
    \\
    \spstsemO{x}{t}{P}
    &= \spspar{\spsbag{\keyed{v_1}{t_1''}, \ldots, \keyed{v_m}{t_m''}}}{r_1'' \cup \cdots \cup r_m''}.
    \quad
\end{align*}
\end{definition}

\section{Examples}
\label{45:ssec:spst-example}

To illustrate the definition of the various SPST constructs,
we continue the example schema from \Cref{45:fig:example-schema} as the input type.
For the output, suppose we want to produce two kinds of events: \code{EndOfHour},
representing end-of-hour summaries, and \code{Outlier}, representing
outlier events that should be logged for further investigation.
We describe building an SPST with this input and output,
building it bottom-up from the structure of the input schema.

We begin with an example of a relational SPST.
We describe the transformation on \code{RideCompleted}
events which computes the sum of the costs of all completed hours.
The interface of this SPST is $
P_1: ((), \relleaf{ \code{RideCompleted} }, \code{float}, \emptyset)$.
As it consumes a bag of \code{RideCompleted} events,
it does not produce any output tuples, but instead we aggregate the sum of the return
costs as a single \code{float}.
For this relational base case,
the computation can be written using an aggregator in a base relational language
such as SQL.
Formally, in our framework $P_1$ is defined by two black-box functions:
$\spstfield{P}{open}((), r) = \spsemp$
and $\spstfield{P}{closed}((), \spsbag{x_1, \ldots, x_m}) = (x_1 + \cdots + x_m, \spsemp)$.
The former component $\spstfield{P}{open}$
indicates that in this case no events are produced incrementally
(as the input stream is processed).
The latter component $\spstfield{P}{closed}$
indicates that the final result of the computation (after the entire input stream is seen)
is the sum of all tuples in the input relation.

Next, we describe a simple sequential SPST
which processes a linear sequence of \code{GPS}
events.
Recall that $\seqleaf{\mathcal{H}}$ is a useful special case of hierarchical synchronization schemas
that denote simple sequences, i.e., it is the schema $\hier{\mathcal{H}}{\relleaf{\varnothing}}$.
Suppose we want to compute the distance traveled for a specific taxi given its \code{GPS} tuples;
additionally,
suppose we want to produce as output \emph{outlier} \code{GPS}
tuples, rather than including them in the aggregation.
\[
P_2: ((),
    \seqleaf{ \code{GPS} },
    \code{float},
    \seqleaf{ \code{Outlier} })
\]


$P_2$ keeps the last known location for the taxi and the current distance traveled as its state, and each time it processes a new \code{GPS} tuple, it updates both.
Additionally, if the last known location is too far from the current one
($> 1$ below)
instead of updating state it produces the tuple as output.
\begin{align*}
\spstfield{P_2}{update}((\bot, 0), gps)
    &= (( gps.loc, 0), \spsemp) \\
\spstfield{P_2}{update}((loc, d), gps)
    &= (( gps.loc, d + dist(gps.loc, loc)) \\
    &\quad\quad\text{if } dist(gps.loc, loc) \le 1 \\
\spstfield{P_2}{update}((loc, d), gps)
    &= (( loc, d), \code{Outlier}(loc)) \\
    &\quad\quad\text{otherwise}
\end{align*}

% \begin{alignat*}{3}
% & \spstfield{P_2}{update}((\bot, 0), gps) && = (( && gps.loc, 0), \spsemp) \\
% & \spstfield{P_2}{update}((loc, d), gps) && = (( && gps.loc, d + \\
% & && && dist(gps.loc, loc)), \spsemp) \\
% & \spstfield{P_2}{update}((loc, d), gps) && = (( && loc, d), \code{Outlier}(loc))
% \end{alignat*}
Because this is a sequential base case
(a special case of hierarchical),
$\spstfield{P_2}{sub}$, $\spstfield{P_2}{call}$, and $\spstfield{P_2}{return}$
are trivial with no effect on the state.
Finally, $\spstfield{P_2}{init}(()) = (\bot, 0)$,
and $\spstfield{P_2}{fin}(loc, d) = (d, \spsemp)$.

Next, we define the partitioned SPST that computes the total distance traveled by all taxis (according to the taxi example described in \Cref{45:fig:example-schema}).
The interface of the SPST is
\[
P_3:
((),
    \keyby{\code{taxiID}}{\seqleaf{ \code{GPS} }},
    \code{float},
    \relleaf{\code{Outlier}})
\]
since it returns the total distance traveled by all taxis in miles.
The child SPST is $P_2$, i.e., $\spstfield{P_3}{sub} = P_2$,
However, notice that instead of a sequential output, here the output outliers
are a bag: this is because there are multiple keys (taxi IDs),
so different key outputs may be unordered.
Implicitly, we are relaxing the output of $P_2$ to be a bag instead of a sequence:
this illustrates SPST \emph{subtyping} (Definition~\ref{45:def:spst-subtyping}),
in which ordered output events may be reinterpreted as unordered.
The interface of our sequential SPST is now
$P_2: \seqleaf{ \code{GPS} }, \code{float}, \seqleaf{ \code{Outlier} })$.
To fit the SPST definition exactly,
we would additionally relax to $\parcomp{\varnothing}{\seqleaf{ \code{Outlier} }}$
(to allow both keyed and bag outputs),
but we leave this off for presentation;
it is just another application of subtyping since the schemas are equivalent.
To complete the definition of $P_3$,
the aggregation
produces a sum of the distances:
\[
\spstfield{P_3}{agg}(\_, ds) = (sum( \{ d \mid (\_, d) \in ds \} ), \spsemp{})
\]
and $\spstfield{P_3}{init}$ initializes all child SPSTs with the unit value.

At this point, we have a partitioned SPST $P_3$ for processing
the key-partitioned \code{GPS} stream,
and we have a relational SPST $P_1$ for processing the
\code{RideCompleted} events.
In order to combine these into
an overall query
which also processes the \code{EndOfHour}
synchronizing events,
we first need to combine these two streams in parallel.
We define an SPST $P_4$ which
divides the aggregate cost by the aggregate distance.
Let $S_1 = \keyby{\code{taxiID}}{\seqleaf{ \code{GPS} }}$
and $S_2 = \relleaf{ \code{RideCompleted} }$.
Then the interface of $P_4$ is
\[
P_4: ((), \parcomp{S_1}{S_2}, \code{float}, \relleaf{ \code{Outlier} }).
\]
The SPST calls the underlying SPSTs $P_1$ and $P_3$: $\spstfield{P_4}{left} = P_3$ and $\spstfield{P_4}{right} = P_1$, which return the total distance covered by all taxis and the total cost of all completed rides in that hour,
and then simply divides to return the \code{float} ride cost per traveled mile, i.e., $\spstfield{P}{fin}(dist, cost) = cost / dist$.

Notice that the average value in $P_4$
is only computed on finalization (after the entire stream is processed).
In order to produce the same averages in a streaming manner,
we need \emph{synchronization events},
and this leads us to our final step:
we complete the input schema in Figure~\ref{45:fig:example-schema}
and the example by constructing a hierarchical schema
which also processes
the \code{EndOfHour} synchronization events.
The schema $P_5$ which outputs the cost per distance traveled
at the end of each hour
has the following interface:
\begin{align*}
P_5: (
    &(), \hier{ \code{EndOfHour} }{ \parcomp{S_1}{S_2} }, \\
    &(), \hier{ \code{CostPerMile} }{ \relleaf{\code{Outlier}} }
)
\end{align*}
The SPST calls the underlying SPST $P_4$, i.e., $\spstfield{P_5}{sub} = P_4$,  which returns the cost per mile in the last hour as a \code{float}.
$P_4$ also produces the \code{Outlier} output events.
The internal state $Y$ is the cost per mile
from the last substream.
The function $\spstfield{P_5}{call}$
does not pass anything to $P_4$,
but $\spstfield{P_5}{return}$ does consume the final \code{float}
and stores it in the state.
Then $P_5$ simply outputs the \code{float} when processing an \code{EndOfHour} tuple:
\[
\spstfield{P_5}{update}(cpm, \_) = (cpm, \code{CostPerMile}(cpm)).
\]

\section{Monotonicity}
\label{45:sec:theorems}

The following is the major technical result of this chapter: for any SPST, the open semantics is monotone in the prefix relation on partially ordered streams.
This ensures that event-by-event incremental processing is possible, though it does not define the event-by-event logic explicitly.

\begin{theorem}
\label{45:thm:spst-monotonicity}
Let $P: (X, S, X', S')$ be an SPST.
Then $P$ is monotone in the following sense:
for any $x: X$ and $t, u: S$,
if $t\prefix u$,
then $\spstsemO{x}{t}{P} \prefix \spstsemO{x}{u}{P}$.
% for any $x: X$ and $t, u: S$,
% there exists $u'$ such that
% $\spstsemO{x}{t \circ u}{P} = \spstsemO{x}{t}{P} \circ u'$.
\end{theorem}
\begin{proof}
  The proof is by induction on $P$.
  We strengthen the hypothesis to additionally show that
  the open semantics is a prefix of the closed semantics:
  if $\spstsemC{x}{t}{P} = (x', t')$ then $\spstsemO{x}{t}{P} \prefix t'$.
  In addition to the definition of concatenation $\circ$
  and prefix $\prefix$,
  we use that $\prefix$ is a partial order
  (Proposition~\ref{45:prop:sps-concat-properties}).
  One of the inductive cases is subtyping
  as given by Definition~\ref{45:def:spst-subtyping}.
  % we use the following derived facts about $\circ$ which can be shown using the definition.
  % (1) Identity: for any SPS $t$, $t \circ \spsemp = \spsemp \circ t = t$,
  % where $\spsemp$ is the empty SPS for that case (empty bag, empty sequence,
  % or empty key-partitioned map).
  % (2) Associativity: $(t_1 \circ t_2) \circ t_3 = t_1 \circ (t_2 \circ t_3)$.
  \begin{itemize}
  \item In the relational case,
  $\spstfield{P}{open}$ is monotonic and a subset relation
  of $\spstfield{P}{closed}$ by assumption.
  \item In the parallel case,
  let $t = \spspar{t_1}{t_2}$ and $u = \spspar{u_1}{u_2}$,
  and suppose that we have
  $\spstsemO{x_1}{t_1}{\spstfield{P}{left}} = t_1'$,
  $\spstsemO{x_1}{u_1}{\spstfield{P}{left}} = u_1'$,
  $\spstsemO{x_2}{t_2}{\spstfield{P}{right}} = t_2'$, and
  $\spstsemO{x_2}{u_2}{\spstfield{P}{right}} = u_2'$.
  Applying the inductive hypothesis, what we need to
  show is that if $t_1' \prefix u_1'$ and $t_2' \prefix u_2'$,
  then
  $\spspar{t_1'}{t_2'} \prefix \spspar{u_1'}{u_2'}$.
  This follows by unfolding the definitions of prefix and underlying concatenation,
  which works component-wise on $\spspar{t_1'}{t_2'}$.
  The same reasoning applies to
  comparing the open and closed semantics.

  \item
  In the hierarchical case,
  our first step is to prove that the auxiliary semantics
  is monotonic.
  For this, we only consider when $t$ and $u$ each end in an empty sub-trace
  $t_m = \spsemp$ and $u_m = \spsemp$.
  This then follows by induction on the trace directly
  since the output is a sequence and produced one item at a time
  from the closed semantics of the sub-SPST,
  using transitivity of $\prefix$.
  Next for the general case,
  we observe the following:
  for any trace ending in an empty subtrace $t$,
  and any subtraces $t_1, u_1$ with $t_1 \prefix u_1$,
  the auxiliary semantics on $t$
  is a prefix of the open semantics on $t \circ [t_1]$
  (by definition),
  which is a prefix of the open semantics on $t \circ [u_1]$
  (by IH),
  which is a prefix of the auxiliary semantics on $t$
  concatenated with closed sub-SPST semantics on $u_1$
  (by definition, IH, and associativity of concatenation),
  which is a prefix of the auxiliary semantics
  on $t \circ [u_1, d, \spsemp{}]$ for any $d$
  (by definition).
  This chain of prefix relations
  implies the general monotonicity for $t \prefix u$,
  using transitivity of $\prefix$.
  Also the auxiliary semantics on $t$ concatenated with closed sub-SPST
  semantics on $u_1$ is a prefix of
  the closed semantics on $t \circ [u_1]$
  (by definition),
  which gives that the open semantics is a prefix of closed.

  \item
  Next we consider the partition case.
  For the open semantics, $\spstfield{P}{agg}$ does not factor in.
  We consider the output on $t \circ u$ and $t$
  in two parts: first the keyed output, and second the relational
  output.
  % \begin{itemize}
  % \item
  (i)
  For the keyed output, we need to show that the output on
  $t$ is a prefix of the output on $t \circ u$.
  There are three cases here: the key is present in both
  $t$ and $u$, present in only $t$, and present in only $u$.
  If present in both, the prefix relation holds by induction hypothesis.
  If only in $t$, the output on $t$ and on $t \circ u$ are the same
  as these SPS are the same for this particular key value.
  If only in $u$, the output on $t$ does not contain this particular key value,
  and so is a prefix of the output on $t \circ u$
  taking $u'$ to be the output on $u$ for that key.
  % \item
  (ii)
  For the relational output, we consider the set of key values
  in $t$: for each such value, the output on $t$ and on $t \circ u$
  produces a relation.
  We can ignore key values not in $t$ (in $t \circ u$ only) as they
  only extend the output relation for $t \circ u$.
  Now we need to show that the relational output on $t \circ u$
  is a superset of the relational output on $t$ for each of these keys,
  which is true by induction hypothesis.
  % \end{itemize}
  \item
  Finally, we consider the case of subtyping (output schema relaxation).
  This requires careful application of
  Definition 13 from~\citeMain{pods21},
  Proposition~\ref{45:prop:sps-sequence-correspondence}
  and Proposition~\ref{45:prop:schema-relaxation-flattening}.
  Using these we derive the following lemma:
  given a schema, $t' \prefix u'$
  is equivalent to the following statement:
  every flattening of $t'$
  can be extended to a flattening of $u'$,
  and every flattening of $u'$ is equivalent
  to a extension of a flattening of $t'$.

  Given this lemma,
  let $S$ and $S'$ be the input and output schemas,
  and $S'' \lesssim S'$.
  Let $t', u', t'', u''$ be the output schemas for $t$ and $u$:
  the definition of $t''$ and $u''$ is that
  all flattenings of $t'$ are flattenings of $t''$,
  and all flattenings of $u'$ are flattenings of $u''$.
  We also know by IH that $t' \prefix u'$,
  which we interpret in terms of flattenings by the lemma.
  Considering any flattening of $t''$,
  first we know it only contains events in $\headers(S')$
  (because the original schema output was $S'$),
  and we can additionally show it is equivalent under
  $S''$
  to some flattening of $t'$;
  this $t'$ then can be extended to a flattening of $u'$,
  so the flattening of $t''$ can be extended with the same extension
  to a flattening of $u''$,
  which by the lemma implies $t'' \prefix u''$.
  \qedhere{}
  \end{itemize}
\end{proof}
