\section{Syntax and Semantics}

This section lays the groundwork for the rest of the technical content of this dissertation: we present our core programmming language for distributed streaming applications. The language is built over a type system for streams called \emph{synchronization schemas}~\citeMain{pods21}, which evolved from earlier work on \emph{data-trace types}~\citeMain{festschrift18,pldi19}. a library of high-level language constructs over typed streams called \emph{SQREL}, and composition rules for composing language constructs.
% dummy citation to please bibtex -- TODO remove
\cite{StreamQRE}

\subsection{Syntax}

\subsubsection{Stream Types}

This definition differs from the stream types in \citeMain{pods21} in a few ways: (1) we don't assume the base types of events are headers (tuples); (2) the synchronization markers are elements of a single base type, not a set of headers; (3) we add the empty stream case, the singleton case, and the add-field case.

\begin{definition}
Let $T$ denote a base type in the following grammar.
A \emph{stream type} is a type defined syntactically by the following grammar:
\[
  S \quad ::= \quad
    \hier{T}{S} \mid
    \parcomp{S}{S} \mid
    \keyby{T}{S} \mid
    \singleton{T} \mid
    \empstream{}
\]
\end{definition}

We also have the following abbreviations:
\[
  \relleaf{T} := \keyby{T}{\singleton{\unittype{}}}
\]
where $\unittype{}$ denotes the unit type, and
\[
  \seqleaf{T} := \hier{T}{\empstream{}}.
\]

\subsubsection{Concrete stream instances}

The following syntax defines concrete stream instances for each of the
stream types.
These are ``batch'' data.
To define a concrete stream instance, one either defines a pair, a sequence, or a bag (unordered set).
\[
  B \quad ::= \quad
    (B, B) \mid
    [B, B, \ldots, B] \mid
    \{B, B, \ldots, B\} \mid
    t: T \mid
    \unittype{}
\]


% \begin{mathpar}
%     \inference[Sequence]
%     {t = d_0 \circ d_1 \cdots d_m \and
%         \forall i, d_i : \tup(\mathcal{H}), \tuprestrict{d_i}{K} = v}
%     {t : \spstype{v}{\seqleaf{\mathcal{H}}}}

%     \inference[Bag]
%     {t = \{ d_0, d_1, \ldots d_m \} \and
%         \forall i, d_i : \tup(\mathcal{H}), \tuprestrict{d_i}{K} = v}
%     {t : \spstype{v}{\relleaf{\mathcal{H}}}}

%     \inference[Synchronization]
%     {t = t_0 \circ d_1 \circ t_1 \cdots d_m \circ t_m \\
%         \forall i, d_i : \tup(\mathcal{H}), \tuprestrict{d_i}{K} = v,  t_i: \spstype{v}{S'}}
%     {t : \spstype{v}{\hier{\mathcal{H}}{S'}}}

%     \inference[PartitionBy]
%     {t = \{ v_1' \mapsto t_1, v_2' \mapsto t_2, \ldots, v_m' \mapsto t_m \} \\
%     v_1', v_2', \ldots, v_m': \tup(K') \text{ distinct};
%     \forall i,
%         v_i: \tup(K \cup K') \text{ s.t.} \\
%         \tuprestrict{v_i}{K'} = v_i',
%         \tuprestrict{v_i}{K} = v,
%         t_i: \spstype{v_i}{S'} \text{ nonempty}
%     }
%     {t : \spstype{v}{\keyby{K'}{S'}}}

%     \inference[Parallel]
%     {t_1 : \spstype{v}{S_1} \and
%      t_2 : \spstype{v}{S_2} \and
%      t = \spspar{t_1}{t_2} }
%     {t : \spstype{v}{\parcomp{S_1}{S_2}}}

% \end{mathpar}
