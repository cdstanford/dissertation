%%% Header for dissertation proposal + dissertation
%%% Caleb Stanford
%%% January 2021 -- Present

%% ========== General ==========

\usepackage{amsmath,amssymb,amsthm}
\usepackage{enumerate}
\theoremstyle{definition}
\newtheorem{theorem}{Theorem}
\newtheorem{lemma}[theorem]{Lemma}
\newtheorem{proposition}[theorem]{Proposition}
\newtheorem{definition}[theorem]{Definition}
\newtheorem{example}[theorem]{Example}
\numberwithin{theorem}{section}

% Inference rules
\usepackage{mathpartir}
\newcommand{\inference}[3][]{\inferrule*[Right=#1]{#2}{#3}}

%% ========== Bibliography ==========

% The following command loads hyperref with back-references, but I don't
% find them super nice. Also the back-references don't work for citeMain.
% \usepackage[backref=page]{hyperref}
\usepackage{hyperref}
\hypersetup{
    colorlinks,
    linkcolor={red!50!black},
    citecolor={blue!50!black},
    urlcolor={blue!80!black}
}

\usepackage{cleveref}
\usepackage{multibib}
\newcites{Main}{Primary References}

%% ========== Document layout ==========

\usepackage[margin=1.2in]{geometry}
\usepackage{parskip}

%% Roman Numerals
\makeatletter
\newcommand*{\romnum}[1]{\romannumeral #1\relax}
\newcommand*{\RomNum}[1]{\expandafter\@slowromancap\romannumeral #1@}
\makeatother

% Fancy custom section headers

\usepackage{changepage}
\usepackage{titlesec}
\titleformat{\section}[display]
  {%
    \clearpage\thispagestyle{empty}%
    \normalfont\Large\bfseries\filcenter\titlerule[1pt]\vskip3pt%
  }
  {\RomNum{\thesection}.}
  {-2pt}
  {}
  [{\vskip5pt\titlerule[1pt]\vspace{0.3cm}}]

\newcommand{\headerblock}[1]{
  \begin{adjustwidth}{1cm}{1cm}
  \begin{center}
    #1
  \end{center}
\end{adjustwidth}
  \vspace{0.3cm}
  % \clearpage
}
\newcommand{\headerimage}[1]{\includegraphics[width=\textwidth]{#1}}
\newcommand{\headerquote}[2]{\textit{#1 \qquad \\ \qquad ---#2}}
\newcommand{\headerbreak}{\vspace{0.3cm}}

%% ========== Graphics ==========

\usepackage[dvipsnames]{xcolor}
\usepackage{graphicx}
\usepackage{graphbox} % allows using [align=c] argument to center vertically
\usepackage{stmaryrd} % Symbols like \llbracket, \rrbracket
\usepackage{pifont} % Symbols like \ding
\usepackage{relsize} % Large symbols (\mathlarger)

\newcommand{\GreenYes}{\color{ForestGreen}{Yes}}
\newcommand{\RedNo}{\color{red}{No}}
\newcommand{\cmark}{\ding{51}}
\newcommand{\xmark}{\ding{55}}

%% ========== Arrays / Tables ==========

\usepackage{verbatim}
\usepackage{algorithm, caption}
\usepackage[noend]{algpseudocode} % (layout for algorithmicx)
\usepackage{subcaption}
\captionsetup{compatibility=false} % Fix compatibility error
\usepackage{listings}

\usepackage{array}
\usepackage{makecell} % Line breaks in tables
% Line break in table cell
% \newcommand{\specialcell}[2][c]{%
% \begin{tabular}[#1]{@{}c@{}}#2\end{tabular}}
% Left/center/right aligned paragraph columns:
\newcolumntype{L}[1]{>{\raggedright\let\newline\\\arraybackslash\hspace{0pt}}m{#1}}
\newcolumntype{C}[1]{>{\centering\let\newline\\\arraybackslash\hspace{0pt}}m{#1}}
\newcolumntype{R}[1]{>{\raggedleft\let\newline\\\arraybackslash\hspace{0pt}}m{#1}}

\newenvironment{Tabular}[2][1] % stretchable tabular
  {\def\arraystretch{#1}\tabular{#2}}
  {\endtabular}

% stack text on top of other text
% usage: \stackanchor{A}{B} or \Shortstack{{a} {b} {c}}
\usepackage{stackengine}

%% ========== TikZ ==========

\usepackage{tikz}
\usetikzlibrary{positioning, matrix, fit, arrows, shapes}
\usetikzlibrary{fit}

%% Block Diagrams
\tikzset{Block/.style={draw, rectangle, inner sep=4pt, align=center}}
\tikzset{Block Edge/.style={draw,thick,->}}
% \tikzset{Data/.append style={ellipse}}
% \tikzset{Input/.append style={fill=red!20}}
% \tikzset{Output/.append style={fill=blue!20}}
\tikzset{Data/.append style={fill=blue!20}}
\tikzset{Input/.append style={}}
\tikzset{Output/.append style={}}

%% Dataflow Graphs
\tikzset{source/.style={
        draw,
        rounded rectangle,
        inner sep=2pt,
        align=center,
        fill=blue!20
}}
\tikzset{operator/.style={
        draw,
        rectangle,
        inner sep=2pt,
        align=center
}}
\tikzset{sink/.style={
        draw,
        rounded rectangle,
        inner sep=2pt,
        align=center,
        fill=blue!20
}}
\tikzset{dataflowedge/.style={draw,thick,->}}

%% Topology Diagrams
\tikzset{Device Node/.style={draw,circle}}
\tikzset{Network/.style={draw,cloud,cloud puffs=10,cloud puff arc=120, aspect=2.5, inner sep=2pt}}
\tikzset{In Edge/.style={->,very thick,red}}
\tikzset{Out Edge/.style={->,very thick,blue}}

% Dependency Relations
\tikzset{Tag Node/.style={draw, circle, inner sep=1pt}}
\tikzset{Tag Edge/.style={draw, thick}}
\tikzset{Tag Loop/.style={draw, thick, in=245, out=305, looseness=5}}
\newcommand{\KeyDepGraph}[1]{
    \begin{tikzpicture}[baseline=(1.base)]
        \node[Tag Node] (1) at (0,0) {\tg{#1}$_1$};
        \draw (1) edge[Tag Loop] (1);
        \node[Tag Node] (2) at (1.2,0) {\tg{#1}$_2$};
        \node[draw=none,fill=none] at (2.1,0) {$\cdots$};
        \draw (2) edge[Tag Loop] (2);
        \node[Tag Node] (3) at (3,0) {\tg{#1}$_k$};
        \node[draw=none,fill=none] at (3.9,0) {$\cdots$};
        \draw (3) edge[Tag Loop] (3);
    \end{tikzpicture}}
\newcommand{\ExtendedKeyDepGraph}[1]{
    \begin{tikzpicture}[baseline=(1.base)]
        \node[Tag Node] (1) at (0,0) {\tg{#1}$_1$};
        \draw (1) edge[Tag Loop] (1);
        \node[Tag Node] (2) at (1.2,0) {\tg{#1}$_2$};
        \node[draw=none,fill=none] at (2.1,0) {$\cdots$};
        \draw (2) edge[Tag Loop] (2);
        \node[Tag Node] (3) at (3,0) {\tg{#1}$_k$};
        \node[draw=none,fill=none] at (3.9,0) {$\cdots$};
        \draw (3) edge[Tag Loop] (3);
        \node[Tag Node] (4) at (1.6,1.2) {\tg{EOD}};
        \draw (4) edge[Tag Loop, in=65, out=115] (4);
        \draw (1) edge[Tag Edge] (4);
        \draw (2) edge[Tag Edge] (4);
        \draw (3) edge[Tag Edge] (4);
        \node[Tag Node] (5) at (0,1.2) {\tg{EOM}};
        \draw (5) edge[Tag Loop, in=65, out=115] (5);
        \draw (4) edge[Tag Edge] (5);
    \end{tikzpicture}}

%% update/fork/join computations
\tikzset{B/.style={draw, inner sep=0pt, circle, font=\footnotesize{}, minimum size=16pt}}
% minimum size=16pt
% regular polygon, regular polygon sides=4,font=\footnotesize}}
\tikzset{E/.style={draw,->}}
\tikzset{W/.style={draw, font=\small}}

\newcommand{\TwoTagDepGraph}[3]{
    \begin{tikzpicture}[baseline=(1.base)]
        \node[Tag Node] (1) at (210:0.5) {#1};
        \node[Tag Node] (2) at (330:0.5) {#2};
        #3
    \end{tikzpicture}}
\newcommand{\ThreeTagDepGraph}[4]{
    \begin{tikzpicture}[baseline=(1.base)]
        \node[Tag Node] (1) at (90:0.3) {#1};
        \node[Tag Node] (2) at (210:0.5) {#2};
        \node[Tag Node] (3) at (330:0.5) {#3};
        #4
    \end{tikzpicture}}
\newcommand{\TopConfigNode}[3]{
    \begin{tabular}{c | c }
        #1 & \{ #2 \} \\
        \hline
        \multicolumn{2}{c}{#3}
    \end{tabular}
}

\newcommand{\ConfigurationNode}[3]{
    node { \TopConfigNode{#1}{#2}{#3} } }

\newcommand{\TopWorkerNode}[3]{

%     \matrix (M) {
%         || #1 \\
%         Independent \\
%         Reduction \\
%         DivideAndConquer \\
%     };
%   \draw[black] (M-2-1.north west) -- (M-2-1.north east);

    \begin{tabular}{c}
        #1 \\
        \hline
        \{ #2 \} \\
        {#3}
    \end{tabular}
}
\newcommand{\TopDNode}[4]{
    % \begin{tabular}{c c c c c c c}
    %     #3 | & & & & & & \\
    %     \hline
    %     \multicolumn{7}{c}{} \\
    %     & \multicolumn{5}{c}{\TopWorkerNode{#1}{#2}{#4}} &
    % \end{tabular}
    \TopConfigNode{#1}{#2}{#4}
}

\newcommand{\DNode}[5]{
    node (#5) { \TopDNode{#1}{#2}{#3}{#4} }
}

%% ========== Code and Pseudocode Customization ==========

%% Listings
\lstset{language=Java,
  showspaces=false,
  showtabs=false,
  breaklines=true,
  showstringspaces=false,
  breakatwhitespace=true,
  commentstyle=\color{gray},
  keywordstyle=\color{blue},
  stringstyle=\color{red},
  basicstyle=\ttfamily\footnotesize,
  frame = single,
  numbersep = 0,
  resetmargins = true
}
\newcommand{\inljava}[1]{\lstinline[columns=fixed, basicstyle=\ttfamily\small]{#1}}

%% Algorithmic
\renewcommand{\algorithmicrequire}{\textbf{Input:}}
\renewcommand{\algorithmicensure}{\textbf{Output:}}
\algblockdefx[OnElement]{OnElement}{EndOnElement}%
  [2]{\textbf{on new element }#1\textbf{ in stream }#2\textbf{:}}%
  {\textbf{done}}
\algblockdefx[OnEmptyStreams]{OnEmptyStreams}{EndOnEmptyStreams}%
  {\textbf{on both streams empty:}}%
  {\textbf{done}}
\makeatletter
\ifthenelse{\equal{\ALG@noend}{t}}%
  {\algtext*{EndOnElement} \algtext*{EndOnEmptyStreams}}%
  {}%
\makeatother

%% For Flumina

\usepackage{tcolorbox}
\newenvironment{takeaway}{
\begin{tcolorbox}[colback=blue!5!white,colframe=blue!5!white,arc=0mm,grow to left by=1.5mm,left=1mm,grow to right by=1.5mm,right=1mm,top=.5mm,bottom=.5mm]
}
{
\end{tcolorbox}
}

\usepackage{colortbl}
\definecolor{ColBlue}{RGB}{227, 252, 252}
\definecolor{ColPink}{RGB}{245, 208, 229}
\definecolor{ColYellow}{RGB}{247, 247, 205}
\newcolumntype{b}{>{\columncolor{ColBlue}}c}
\newcolumntype{f}{>{\columncolor{ColYellow}}c}
\newcolumntype{t}{>{\columncolor{ColPink}}c}

%% ========== Listing Config  ==========

\usepackage{listings}
% \usepackage{lstlinebgrd}

\lstset{
    numbers=none,
    % numberstyle=\tiny\color{black},
    basicstyle=\ttfamily\footnotesize,
    basewidth=0.59em,
    keywordstyle=[3]{},
    commentstyle=\itshape\footnotesize,
    tabsize=8,
    % frame=single,
    % frameround=tttt,
    showstringspaces=false,
    breaklines=false,
    captionpos=b,
    aboveskip=\bigskipamount,
    belowskip=\bigskipamount,
    escapechar=^
}

%% Erlang Code

\lstdefinestyle{custom}{
    numbers=none,
    % numberstyle=\tiny\color{black},
    basicstyle=\ttfamily\footnotesize,
    basewidth=0.59em,
    keywordstyle=[3]{},
    commentstyle=\itshape\footnotesize,
    tabsize=8,
    % frame=single,
    % frameround=tttt,
    showstringspaces=false,
    rulecolor=\color{black},
    breaklines=false,
    captionpos=b,
    aboveskip=\bigskipamount,
    belowskip=\bigskipamount,
    escapechar=^,
    moredelim=**[is][\color{blue}]{@}{@}
}
%
% \lstnewenvironment{ErlangCode}[2]{
%   \nopagebreak
%   \lstset{language=Erlang, morekeywords={spec,number},
%     label={#1}, caption={#2}, style=custom}
% }{}

\crefname{lstlisting}{listing}{listings}
\Crefname{lstlisting}{Listing}{Listings}

\newcommand{\inle}[1]{\lstinline[columns=fixed,language=Erlang,style=custom]{#1}}

%% Table coloring
\definecolor{ColBlue}{RGB}{227, 252, 252}
\definecolor{ColPink}{RGB}{245, 208, 229}
\definecolor{ColYellow}{RGB}{247, 247, 205}
\newcolumntype{b}{>{\columncolor{ColBlue}}c}
\newcolumntype{f}{>{\columncolor{ColYellow}}c}
\newcolumntype{t}{>{\columncolor{ColPink}}c}

% \renewcommand{\paragraph}[1]{\vspace{2pt}\noindent\textbf{#1}\enspace}

%% Flumina code
\definecolor{PigBlue}{RGB}{30, 0, 200}
\definecolor{PigRed}{RGB}{150, 0, 0}
\definecolor{PigGreen}{RGB}{0, 128, 0}
\lstdefinelanguage{Flumina}
{
    keywords=[1]{
        if, else, for, in, return, output, new, match, with
    },
    keywordstyle=[1]\bfseries,
    keywords=[2]{
        init,
        dependencies,
        fork, join,
        update, update_i, update_s, update_o,
        out, out_i,
        pred_i, pred_0, pred_j
    },
    keywordstyle=[2]\color{PigRed},
    keywords=[3]{
        Map, Set, List,
        Event, State, Integer, Key, Out, Tag, Payload, Heartbeat,
        State_i, State_j, State_k, State_0, State_1, State_2,
        State1, State2,
        Pred,
        List
    },
    keywordstyle=[3]\color{PigBlue},
    sensitive=true,
    morestring=[b]',
    morecomment=[l][\color{black!70}]{//},
    basicstyle=\scriptsize\ttfamily
    % basicstyle=\small %, %or \small or \footnotesize etc.]
}

\lstnewenvironment{FluminaCode}{
  \nopagebreak
  \lstset{language=Flumina}}{}

\newcommand{\fl}[1]{\text{\lstinline[
    language=Flumina,
    columns=fullflexible,
    basicstyle=\footnotesize\ttfamily
]!#1!}}
