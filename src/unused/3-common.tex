\subsection{Synchronization Schemas}

We argue that streams should be viewed as \emph{partially ordered},
and ordering requirements should be defined by using
\emph{types} to describe these partial orders on concrete streams.

Why ``schemas''? A table in a database is typically defined using a \emph{relational schema}, which defines the type of items (tuples) in the relation. By analogy, a stream in a stream processing system can be defined by a synchronization schema, which defines the type of items in the stream \emph{in addition to} a binary relation between items called \emph{synchronizes}: ``$a$ synchronizes $b$'' means that $a$ and $b$ are comparable in the partial order ($a < b$ or $b < a$), subject to an additional restriction that if both $a$ and $a'$ synchronize $b$, then $a$ and $a'$ are also comparable ($a < a'$ or $a' < a$).
% TBD
We first formalize this idea, as it is the crux of what a synchronization schema denotes.

\begin{definition}
Let $D$ be a set of data items. A \emph{synchronization relation} on $D$ is a binary relation $\preceq$ subject to three conditions:
\begin{enumerate}[(1)]
\item Transitivity: if $a \preceq b$ and $b \preceq c$ then $a \preceq c$.
\item If $a \preceq b$ and $a' \preceq b$, then either $a \preceq a'$ or $a' \preceq a$.
\end{enumerate}
\end{definition}
